% This is a template for doing homework assignments in LaTeX

\documentclass{article} % This command is used to set the type of document you are working on such as an article, book, or presenation

% \usepackage{geometry} % This package allows the editing of the page layout
\usepackage{geometry}
\geometry{
 a4paper,
 total={170mm,257mm},
 left=20mm,
 top=20mm,
 right=20mm,
 bottom=20mm,
 }
\usepackage{amsmath}  % This package allows the use of a large range of mathematical formula, commands, and symbols
% \usepackage{graphicx}  % This package allows the importing of images
\usepackage[american]{circuitikz}% This package allows the use of circuits, siunitx is used for labels



\newcommand{\question}[2][]{\begin{flushleft}
    \textbf{Question #1}: \textit{#2}

\end{flushleft}}
\newcommand{\sol}{\textbf{Solution}:} %Use if you want a boldface solution line
\newcommand{\maketitletwo}[2][]{\begin{center}
    \Large{\textbf{Assignment #1}
      
      Circuits I} % Name of course here
    \vspace{5pt}
    
    \normalsize{}        % Change to due date if preferred
    \vspace{15pt}
    
  \end{center}}
\begin{document}
\maketitletwo[1]  % Optional argument is assignment number
% Keep a blank space between maketitletwo and \question[1]


\question[]{Use the following diagram to Answer Question \textit{1}, \textit{2} and \textit{3} }
\begin{figure}[h]\centering
\begin{circuitikz} \draw
% Left side
(0,0) to node[ground]{} ++(0,0)
(0,0)
    to[I,l=$2A$,invert] ++(0,3)
    to[short] ++(3,0)              node[coordinate](V1){}
    to[R,l=$1\Omega$,*-*] ++ (3,0) node[coordinate](V2){}
    to[R,l=$3\Omega$,*-*] ++ (3,0) node[coordinate](V3){}
    to[short] ++(3,0)
    to[R,l=$2\Omega$] ++ (0,-3)
    to[short] ++(-12,0)

(V1) to[R,l=$2\Omega$] ++ (0,-3)
(V2) to[R,l=$4\Omega$] ++ (0,-3)
(V3) to[I,l=$1A$,invert] ++ (0,-3)

(V1) node[above]{$V_{1}$}
(V2) node[above]{$V_{2}$}
(V3) node[above]{$V_{3}$}
;
% \draw[thin, <-, >=triangle 45] (1.5,2.5)node{$i_1$}  ++(-60:0.5) arc (-60:170:0.5);
% \draw[thin, <-, >=triangle 45] (4.5,2.5)node{$i_2$}  ++(-60:0.5) arc (-60:170:0.5);
% \draw[thin, <-, >=triangle 45] (7.5,2.5)node{$i_3$}  ++(-60:0.5) arc (-60:170:0.5);
% \draw[thin, <-, >=triangle 45] (4.5,6.0)node{$i_4$}  ++(-60:0.5) arc (-60:170:0.5);

\end{circuitikz}
\end{figure}

\question[1]{Write the G Matrix}

\vspace{30ex}

\question[2]{Write the R Matrix}

\vspace{30ex}

\question[3]{Find the Power loses in $R_{4}$ ($R_{4}=4\Omega$)}


\end{document}
