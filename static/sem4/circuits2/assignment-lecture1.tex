% This is a template for doing homework assignments in LaTeX

\documentclass{article} % This command is used to set the type of document you are working on such as an article, book, or presenation

% \usepackage{geometry} % This package allows the editing of the page layout
\usepackage{geometry}
\geometry{
 a4paper,
 total={170mm,257mm},
 left=20mm,
 top=20mm,
 right=20mm,
 bottom=20mm,
 }
\usepackage{amsmath}  % This package allows the use of a large range of mathematical formula, commands, and symbols
% \usepackage{graphicx}  % This package allows the importing of images
\usepackage[american]{circuitikz}% This package allows the use of circuits, siunitx is used for labels



\newcommand{\question}[2][]{\begin{flushleft}
    \textbf{Question #1}: \textit{#2}

\end{flushleft}}
\newcommand{\sol}{\textbf{Solution}:} %Use if you want a boldface solution line
\newcommand{\maketitletwo}[2][]{\begin{center}
    \Large{\textbf{Assignment #1}
      
      Circuits I} % Name of course here
    \vspace{5pt}
    
    \normalsize{}        % Change to due date if preferred
    \vspace{15pt}
    
  \end{center}}
\begin{document}
\maketitletwo[1]  % Optional argument is assignment number
% Keep a blank space between maketitletwo and \question[1]


%\question[1]{Use the following diagram to Answer Question \textit{2} and \textit{3} } 
\begin{figure}[h]\centering
\begin{circuitikz} \draw
% Left side
  (0,0) to[V,l=$10V$,invert] ++(0,3)
        to[R,l=$2\Omega$] ++(0,2) node[coordinate](UPPER-LEFT){}
        to[R,l=$4\Omega$] ++(3,0) node[coordinate](ABOVE-6OHM){}
        to[short] ++(3,0) node[coordinate](ABOVE-10OHM){}
        to[R,l=$16\Omega$] ++(3,0)
        to[R,l=$20\Omega$] ++(0,-2)
        to[V,l=$20V$]      ++(0,-3)
        to[R,l=$4\Omega$]      ++(-9,0)
 (UPPER-LEFT)
        to[short] ++(0,2)
        to[R,l=$2\Omega$] ++(4.5,0)
        to[V,l=$15V$] ++(4.5,0)
        to[R,l=$8\Omega$] ++(0,-2)
(ABOVE-6OHM)
        to[R,l=$6\Omega$] ++(0,-5)
        to node[ground]{} ++(0,0)
(ABOVE-10OHM)
        to[R,l=$10\Omega$] ++(0,-5)
        
;
\draw[thin, <-, >=triangle 45] (1.5,2.5)node{$i_1$}  ++(-60:0.5) arc (-60:170:0.5);
\draw[thin, <-, >=triangle 45] (4.5,2.5)node{$i_2$}  ++(-60:0.5) arc (-60:170:0.5);
\draw[thin, <-, >=triangle 45] (7.5,2.5)node{$i_3$}  ++(-60:0.5) arc (-60:170:0.5);
\draw[thin, <-, >=triangle 45] (4.5,6.0)node{$i_4$}  ++(-60:0.5) arc (-60:170:0.5);

\end{circuitikz}
\end{figure}


\end{document}

